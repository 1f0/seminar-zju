\documentclass{beamer}
\usetheme{Antibes}

\usepackage[makeroom]{cancel}

\title[FastWN]{Fast Winding Numbers for Soups and Clouds}
\author{Gavin Braill (University of Toronto), Neil G.Dickson, Ryan Schmidt, David I.W. Levin and Alec Jacobson\\TOG August 2018}
\date{\tiny\today}

% control sequence whose name is non-letter
% doesn't require either spaces or braces aftet it
\def\*#1{\mathbf{#1}}
\newcommand{\norm}[1]{\left\lVert#1\right\rVert}

\begin{document}
\frame{\titlepage}

% \begin{frame}
%   \frametitle{Outline}
%   \tableofcontents
% \end{frame}

\frame{{Determining whether a point is inside or outside}
  \includegraphics<1>[width=0.7\textwidth]{img/use}
  \begin{itemize}
    \item<2> Challenges
    \begin{itemize}
      \item<2> Open boundary
      \item<2> Degenerate geometry
      \item<2> Self-intersections
      \item<2> Non-manifold
      \item<2> Fast Enough!
    \end{itemize}
  \end{itemize}
  
}

\frame{{What is Winding Number?}
  \begin{itemize}
    \item<1> For 2D curve under polar coordinate, $\frac{\theta(1)-\theta(0)}{2\pi}$
    \item<3> For suface, $w_S(\*q)=\frac{1}{4\pi}\int_S d\Omega(\*q)$, $d\Omega$ is solid angle.
  \end{itemize}
  \includegraphics<1>[width=0.6\textwidth]{img/wn}
  \includegraphics<2>[width=\textwidth]{img/harm}
}

\frame{{For Point Cloud}
  \begin{itemize}
    \item $w_S(\*q) \approx \sum_{i=1}^{m}a_i\frac{(\*p_i-\*q)\cdot \*n_i}{4\pi||\*p_i-\*q||^3}$
    \item $a_i$ is the geodesic Voronoi area of the point on the surface.
    %\item $a_i$ is approximated by projection of k-nearest neighbors onto the best-fit plane. Their method are not sensitive to area accuracy.
  \end{itemize}
  \includegraphics[width=0.7\textwidth]<1>{img/ts}
  %\includegraphics[width=\textwidth]<2>{img/sen}
}

\frame{{Fast Approximation}
  \begin{itemize}
    \item<1> Naive method: n query, m points, complexity: $\mathcal{O}(nm)$
    \item<2> Approximation: use octree or AABB tree for partion the space
  \end{itemize}
  \includegraphics<1>[width=\textwidth]{img/appr}
  \includegraphics<2>[width=\textwidth]{img/far}
}

\frame{{Taylor Expansion}
  \begin{itemize}
    \item $\nabla G(\*q,\tilde{\*p}) = \frac{(\tilde{\*p}-\*q)}{4\pi||\tilde{\*p}-\*q||^3}$
    \item $\tilde{\*p} = \sum a_i \*p_i$
    \item<1> For point cloud
    \item<2> For triangle soups
\end{itemize}
  \includegraphics<1>[width=0.8\textwidth]{img/taylor}
  \includegraphics<2>[width=0.8\textwidth]{img/taylor2}
}

\frame{{Experiments and Application}
  \begin{itemize}
    \item<1> Test on \textit{Thingi10k}
    \item<2> Test on different $\beta$.
  \end{itemize}
  \includegraphics<1>[width=\textwidth]{img/thingi}
  \includegraphics<2>[width=0.5\textwidth]{img/beta}
}

\frame{{Application}
  \begin{itemize}
    \item<1> Boolean operation.
    \item<2> Extract $\frac{1}{2}$ iso-value surface
    \item<3> Directly output \textit{toolpath} for 3D print
    \item<4> Voxelization for skining weight
  \end{itemize}
  \includegraphics<1>[width=0.5\textwidth]{img/bool}
  \includegraphics<2>[width=0.6\textwidth]{img/iso}
  \includegraphics<3>[width=\textwidth]{img/print}
  \includegraphics<4>[width=0.8\textwidth]{img/maya}
}

% \frame{{Related Works}
%   \begin{itemize}
%     \item Fast Multipole Method
%     \item Poisson Surface Reconstruction
%     \item Alec Jacobson \textit{Robust Inside-Outside Segmentation using Generalized Winding Numbers} [2013]
%   \end{itemize}
% }

\end{document}

